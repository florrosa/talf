\documentclass{article}

\usepackage[spanish,activeacute]{babel}
\usepackage{amsfonts}

\title{Taller 1}
\author{Mónica López Pola}

\begin{document}

\maketitle

\section*{Enunciado}

\textbf{Una expresión regular es ambigua si una cadena puede ser generada de dos formas diferentes a partir de la misma expresión regular. Especifica si las siguientes expresiones regulares sobre el alfabeto {a,b,c,d} son ambiguas o no, y explica claramente el por qué (razona qué lenguaje generan y demuestra o refuta adecuadamente tu respuesta):}

\textbf{a)}
\textit{a((ab)*cd)* + a(ababcb*)a*}

\textbf{b)}
\textit{aab*(ab)* + ab* + a*bba*}

\textbf{c)}
\textit{aaba* + aaaba + aabba* + a}

\section*{Resolución}

a) Esta expresión genera cadenas que empiezan por una <<a>> seguida de $n$ cadenas formados por la repetición de la cadena <<ab>> $m$ veces y la cadena <<cd>> una vez, y cadenas formadas por una <<a>> seguida de una cadena formada por la repetición de la cadena <<ababcb>> $k$ veces y la cadena formada por el caracter <<a>> repetido $v$ veces.
Siendo $m, n, k, v\in\mathbb{N}_{0}$.

Además, la expresión es ambigua porque la cadena <<a>> se puede generar de dos formas distintas.

\hspace{1cm}

b) Esta expresión genera cadenas que empiezan por <<aa>>, segido por $n$ veces la cadena <<b>> y $m$ veces la cadena <<ab>>; cadenas que empiezan por <<a>> seguido por $p$ veces la cadena <<b>>; y cadenas que empiezan por $q$ veces la cadena <<a>>, seguido por la cedena <<bb>> y $k$ veces la cadena <<a>>. Siendo $m, n, p, q, k\in\mathbb{N}_{0}$.

Además, es ambigua porque las cadenas <<aabb>> y <<abb>> se pueden generar de dos formas distintas cada una.

\hspace{1cm}

c) Esta expresión generas cadenas que empiezan por <<aab>> o <<aabb>>, seguido por $m$ veces la cadena <<a>>; la cadena <<aaaba>>; y la cadena <<a>>. Siendo $m\in\mathbb{N}_{0}$.

Esta cadena no presenta ninguna ambiguedad.

\end{document}